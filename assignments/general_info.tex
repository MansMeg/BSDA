
\subsubsection*{General information}


\begin{itemize}
\item The recommended tool in this course is R (with the IDE R-Studio). You can download R \href{https://cran.r-project.org/}{\textbf{here}} and R-Studio \href{https://www.rstudio.com/products/rstudio/download/}{\textbf{here}}. There are many tutorials, videos and introductions to R and R-Studio online. You can find some initial hints from \href{https://education.rstudio.com/}{\textbf{RStudio Education pages}}.
\item When working with R, we recommend writing the report using R markdown and the provided \href{https://github.com/MansMeg/BSDA/blob/main/templates/assignment_template.rmd}{\textbf{R markdown template}}.
The template includes the formatting instructions and how to include code and figures.
\item Instead of R markdown, you can use other software to make the PDF report, but you should use the same instructions for formatting. These instructions are also available in \href{https://github.com/MansMeg/BSDA/blob/main/templates/assignment_template.pdf}{\textbf{the PDF produced from the R markdown template}}.
\item We supply a Google Colab notebook that you can also use for the assignments. We have included the installation of all necessary R packages; hence, this can be an alternative to using your own local computer. You can find the notebook \href{https://github.com/MansMeg/BSDA/blob/main/templates/bsda_colab_template.ipynb}{\textbf{here}}. You can also open the notebook in Colab \href{https://colab.research.google.com/github/MansMeg/BSDA/blob/main/templates/bsda_colab_template.ipynb}{\textbf{here}}.
\item Report all results in a single and \emph{anonymous} pdf. Note that no other formats are allowed.
\item The course has its own R package \texttt{bsda} with data and functionality to simplify coding. To install the package, just run the following (upgrade="never" skips question about updating other packages):
\begin{enumerate}
\item \texttt{install.packages("remotes")}
\item \texttt{remotes::install\_github("MansMeg/BSDA", \\ subdir = "rpackage", upgrade="never")}
\end{enumerate}
\item Many of the exercises can be checked automatically using the R package \\ \texttt{markmyassignment}. you can find information on how to install and use the package \href{https://cran.r-project.org/web/packages/markmyassignment/vignettes/markmyassignment.html}{\textbf{here}}. There is no need to include \texttt{markmyassignment} results in the report.
\item You can find common questions and answers regarding the installation and technical problems in \href{https://github.com/MansMeg/BSDA/blob/main/FAQ.md}{Frequently Asked Questions (FAQ)}.
\item You can find deadlines and information on how to turn in the assignments in Studium.
\item You are allowed to discuss assignments with your friends, but it is not permitted to copy solutions directly from other students or the internet. Try to solve the actual assignment problems with your code and explanations. Do not share your answers publicly. We compare the answers with the "urkund" system. We will report all suspected plagiarism.
\item If you have any suggestions or improvements to the course material, please post in the course chat feedback channel, create an issue, or submit a pull request to the public repository \href{https://github.com/MansMeg/BSDA/issues}{\textbf{here}}.
\item It is \emph{mandatory} to include the following parts in all assignments (these are included already in the template):
\begin{enumerate}
\item Time used for reading: How long time took the reading assignment (in hours)
\item Time used for the assignment: How long time took the basic assignment (in hours)
\item Good with assignment: Write one-two sentences of what you liked with the assignment/what we should keep for next year.
\item Things to improve in the assignment: Write one-two sentences of what you think can be improved in the assignment. Can something be clarified further? Did you get stuck on stuff unrelated to the content of the assignment etc.
\end{enumerate}
\item You can find information on how each assignment will be graded and how points are assigned \href{https://github.com/MansMeg/BSDA/tree/main/grading}{\textbf{here}}. \textbf{Note!} This grading information can change during the course, for example, if we find errors or inconsistencies. Please feel free to comment on these grading instructions, ideally before turning in your assignment, if you think something is missing or is incorrect.
\item To pass (G) the assignment, you need 70\% of the total points. To pass with distinction (VG), you need 90\% of the total points. See the grading information on the point allocations for each assignment.
\item You are not allowed to show your assignments (text or code) to anyone. Only discuss the assignments with your fellow students. The student that show their assignment to anyone else could also be considered to cheat. Similarly, on zoom labs, only screen share when you are in a separate zoom room with teaching assistants.
\end{itemize}


