\documentclass[10pt,handout]{beamer}
%\documentclass[10pt]{beamer}
\usepackage[english]{babel} % Anpassa efter svenska. Ger svensk logga.
\usepackage[utf8]{inputenc} % Anpassa efter linux
\usepackage{graphicx}
\usepackage{hyperref}
\usepackage{listings}

\hypersetup{
    colorlinks=true,
    linkcolor=blue,
    filecolor=magenta,
    urlcolor=cyan,
}
\usepackage{../common/beamerthemeUppsala}
%\usetheme{Uppsala}
%\usecolortheme{UU} % Anpassa efter UU:s frger och logga
%\hypersetup{pdfpagemode=FullScreen} % Adobe Reader ska ppna fullskrm
\setbeamertemplate{itemize items}[circle]

% \usepackage{beamerthemesplit}
\usepackage{amsmath}
% \usepackage{amssymb}
% \usepackage{graphics}
% \usepackage{graphicx}
% \usepackage{epsfig}
% \usepackage[latin1]{inputenc}
 \usepackage{color}
% \usepackage{fancybox}
% \usepackage{psfrag}
% \usepackage[english]{babel}
 \setbeamertemplate{footline}{\hfill\insertframenumber/\inserttotalframenumber}

% Input new commands
%\usepackage{bm}
%\usepackage{natbib}
\newcommand{\bfm}[1]   {\mbox{\boldmath{${#1}$}}}
\newcommand{\Prob}   {\mbox{\textnormal{P}}}
\def\eqd{\,{\buildrel d \over =}\,}

% Vector/Matrix definitions (in bold type)
\newcommand{\vect}[1]{\mathbf{#1}}
\newcommand{\vectb}[1]{\bm{#1}}

% Differential operator 'd' as upright as in (use \dd)
\newcommand{\dd}{\; \mathrm{d}}

% Gaussian normal distribution (use \N)
\newcommand{\N}{\mathcal{N}} %% or \mathrm{N}

% Uniform distribution (use \Uni)
\newcommand{\Uni}{\mathcal{U}} %% or \mathrm{U}

% Matrix transpose (use \T)
\newcommand{\T}{^{\mathsf{T}}}

% Blockdiagonal matrices (use \blockdiag)
\newcommand{\blockdiag}{\mathrm{blockdiag}}

% Define inner product '<f,g>' notation (use \innerp{#1})
\providecommand{\innerp}[1]{\left\langle#1\right\rangle}

\def\o{{\mathbf o}}
\def\t{{\mathbf \theta}}
\def\w{{\mathbf w}}
\def\x{{\mathbf x}}
\def\y{{\mathbf y}}
\def\z{{\mathbf z}}



% Other math symbols and notation
\newcommand{\D}{^\mathsf{\dagger}}
\newcommand{\R}{\mathbb{R}}
\newcommand{\erf}{\mathrm{erf}}
\newcommand{\E}{\mathrm{E}}
\newcommand{\var}{\mathrm{var}}
\newcommand{\Var}{\mathrm{Var}}
\newcommand{\cov}{\mathrm{cov}}
\newcommand{\Ker}{\operatorname{Ker}}
\newcommand{\Ran}{\operatorname{Ran}}
\providecommand{\norm}[1]{\lVert#1\rVert}
\providecommand{\op}[1]{\mathcal{#1}}
\newcommand{\arccot}{\mathrm{arccot}}
\providecommand{\Hspace}[1]{\mathscr{#1}}
\providecommand{\fourier}[1]{\mathscr{#1}}

\newcommand{\kin}{k^{\rm in}}
\newcommand{\kout}{k^{\rm out}}
\newcommand{\gi}{{R_0}}
\newcommand{\eff}{{E_{\rm max}}}
\newcommand{\HN}{{\rm N^+}}
\newcommand{\lN}{{\rm LN}}

\DeclareMathOperator{\Sd}{Sd}
\DeclareMathOperator{\sd}{sd}
\DeclareMathOperator{\Gammad}{Gamma}
\DeclareMathOperator{\Invgamma}{Inv-gamma}
\DeclareMathOperator{\Bin}{Bin}
\DeclareMathOperator{\Negbin}{Neg-bin}
\DeclareMathOperator{\Poisson}{Poisson}
\DeclareMathOperator{\Beta}{Beta}
\DeclareMathOperator{\logit}{logit}
\DeclareMathOperator{\BF}{BF}
\DeclareMathOperator{\Invchi2}{Inv-\chi^2}
\DeclareMathOperator{\NInvchi2}{N-Inv-\chi^2}
\DeclareMathOperator{\InvWishart}{Inv-Wishart}
\DeclareMathOperator{\tr}{tr}
% \DeclareMathOperator{\Pr}{Pr}
\def\euro{{\footnotesize \EUR\, }}
\DeclareMathOperator{\rep}{\mathrm{rep}}


%%%%%%%%%%%%%%%%%%%%%%%%%%%%%%%%%%%%%%%%%%%%%%%%%%%%%%%%%%%%%%%%%%

\setlength{\parskip}{3mm}
\title[]{{\color{black}Bayesian Statistics and Data Analysis \\ Course information}}
\author[]{M{\aa}ns Magnusson \\ Department of Statistics, Uppsala University}
\date{}

\begin{document}

\frame{\titlepage
% \thispagestyle{empty}
}

%%%%%%%%%%%%%%%%%%%%%%%%%%%%%%%%%%%%%%%%%%%%%%%%%%%%%%%%%%%%%%%%%%

%%%%%%%%%%%%%%%%%%%%%%%%%%%%%%%%%%%%%%%%%%%%%%%%%%%%%%%%%%%%%%%%%%

\section{Course information}
\frame{\sectionpage}

\begin{frame}{Course information}
The aims of this course are that you should:\\[3mm]\pause
\begin{enumerate}
\item get a good knowledge of a large number of machine learning models,
\item become able to use methods for evaluating and improving predictive models,
\item become able to handle big data,
\item become able to train and use machine learning models in R,
\item become able to train and use neural networks using Keras/TensorFlow.
\item become able to describe and discuss ethical aspects of big data and black box-models,
\end{enumerate}

\end{frame}


\begin{frame}
  \frametitle{Pre-requisites}
  \begin{itemize}
  \item Basic terms of probability theory
    \begin{itemize}
    \item probability, probability density, distribution
    \item sum, product rule, and Bayes' rule
    \item expectation, mean, variance, median
  \end{itemize}
  \item Some linear algebra and calculus
  \item Basic visualisation techniques (R or Python)
  \begin{itemize}
    \item histogram, density plot, scatter plot
  \end{itemize}
  \end{itemize}

  First assignment is a recap.

\end{frame}



\begin{frame}{Course Outline}
Two main parts:
\begin{itemize}
\item Core Content (9 lecture blocks)\pause
\item Assignments (8 individual assignments)\pause
\item Mini-project on a supervised project (2-3 students)\pause
\end{itemize}
Exact dates and details; see the course page.
\end{frame}

\begin{frame}{Core Content}

\begin{itemize}
\item Two lectures/computer labs (approx. 4h)
\begin{itemize}
\item Lecture(s): present overall theory and content (overview)
\item Computer labs(s): Hands on help with the assignment if you get stuck. \emph{Start before the computer assignment!}
\end{itemize}
\item Online video material and reading assignments (approx. 4-6h, 50-90 pages a week)
\item \emph{Note!} There might be some overlap between reading instructions.
\item An individual computer assignment (approx. 12-16h). Deadline Sundays 23.59.\pause
\item Recommended workflow for each week
\begin{itemize}
\item Do the reading assignments
\item Attend lecture
\item Watch the videos (although, optional)
\item Do the assignment
\end{itemize}
\end{itemize}

\end{frame}




\end{document}


