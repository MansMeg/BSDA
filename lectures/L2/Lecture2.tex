\documentclass[10pt,handout]{beamer}
%\documentclass[10pt]{beamer}
\usepackage[english]{babel} % Anpassa efter svenska. Ger svensk logga.
\usepackage[utf8]{inputenc} % Anpassa efter linux
\usepackage{graphicx}
\usepackage{hyperref}
\usepackage{listings}

\hypersetup{
    colorlinks=true,
    linkcolor=blue,
    filecolor=magenta,
    urlcolor=cyan,
}
\usepackage{../common/beamerthemeUppsala}
%\usetheme{Uppsala}
%\usecolortheme{UU} % Anpassa efter UU:s frger och logga
%\hypersetup{pdfpagemode=FullScreen} % Adobe Reader ska ppna fullskrm
\setbeamertemplate{itemize items}[circle]

% \usepackage{beamerthemesplit}
\usepackage{amsmath}
% \usepackage{amssymb}
% \usepackage{graphics}
% \usepackage{graphicx}
% \usepackage{epsfig}
% \usepackage[latin1]{inputenc}
 \usepackage{color}
% \usepackage{fancybox}
% \usepackage{psfrag}
% \usepackage[english]{babel}
 \setbeamertemplate{footline}{\hfill\insertframenumber/\inserttotalframenumber}

% Input new commands
%\usepackage{bm}
%\usepackage{natbib}
\newcommand{\bfm}[1]   {\mbox{\boldmath{${#1}$}}}
\newcommand{\Prob}   {\mbox{\textnormal{P}}}
\def\eqd{\,{\buildrel d \over =}\,}

% Vector/Matrix definitions (in bold type)
\newcommand{\vect}[1]{\mathbf{#1}}
\newcommand{\vectb}[1]{\bm{#1}}

% Differential operator 'd' as upright as in (use \dd)
\newcommand{\dd}{\; \mathrm{d}}

% Gaussian normal distribution (use \N)
\newcommand{\N}{\mathcal{N}} %% or \mathrm{N}

% Uniform distribution (use \Uni)
\newcommand{\Uni}{\mathcal{U}} %% or \mathrm{U}

% Matrix transpose (use \T)
\newcommand{\T}{^{\mathsf{T}}}

% Blockdiagonal matrices (use \blockdiag)
\newcommand{\blockdiag}{\mathrm{blockdiag}}

% Define inner product '<f,g>' notation (use \innerp{#1})
\providecommand{\innerp}[1]{\left\langle#1\right\rangle}

\def\o{{\mathbf o}}
\def\t{{\mathbf \theta}}
\def\w{{\mathbf w}}
\def\x{{\mathbf x}}
\def\y{{\mathbf y}}
\def\z{{\mathbf z}}



% Other math symbols and notation
\newcommand{\D}{^\mathsf{\dagger}}
\newcommand{\R}{\mathbb{R}}
\newcommand{\erf}{\mathrm{erf}}
\newcommand{\E}{\mathrm{E}}
\newcommand{\var}{\mathrm{var}}
\newcommand{\Var}{\mathrm{Var}}
\newcommand{\cov}{\mathrm{cov}}
\newcommand{\Ker}{\operatorname{Ker}}
\newcommand{\Ran}{\operatorname{Ran}}
\providecommand{\norm}[1]{\lVert#1\rVert}
\providecommand{\op}[1]{\mathcal{#1}}
\newcommand{\arccot}{\mathrm{arccot}}
\providecommand{\Hspace}[1]{\mathscr{#1}}
\providecommand{\fourier}[1]{\mathscr{#1}}

\newcommand{\kin}{k^{\rm in}}
\newcommand{\kout}{k^{\rm out}}
\newcommand{\gi}{{R_0}}
\newcommand{\eff}{{E_{\rm max}}}
\newcommand{\HN}{{\rm N^+}}
\newcommand{\lN}{{\rm LN}}

\DeclareMathOperator{\Sd}{Sd}
\DeclareMathOperator{\sd}{sd}
\DeclareMathOperator{\Gammad}{Gamma}
\DeclareMathOperator{\Invgamma}{Inv-gamma}
\DeclareMathOperator{\Bin}{Bin}
\DeclareMathOperator{\Negbin}{Neg-bin}
\DeclareMathOperator{\Poisson}{Poisson}
\DeclareMathOperator{\Beta}{Beta}
\DeclareMathOperator{\logit}{logit}
\DeclareMathOperator{\BF}{BF}
\DeclareMathOperator{\Invchi2}{Inv-\chi^2}
\DeclareMathOperator{\NInvchi2}{N-Inv-\chi^2}
\DeclareMathOperator{\InvWishart}{Inv-Wishart}
\DeclareMathOperator{\tr}{tr}
% \DeclareMathOperator{\Pr}{Pr}
\def\euro{{\footnotesize \EUR\, }}
\DeclareMathOperator{\rep}{\mathrm{rep}}


%%%%%%%%%%%%%%%%%%%%%%%%%%%%%%%%%%%%%%%%%%%%%%%%%%%%%%%%%%%%%%%%%%

\setlength{\parskip}{3mm}
\title[]{{\color{black}Bayesian Statistics and Data Analysis \\ Lecture 1}}
\author[]{M{\aa}ns Magnusson \\ Department of Statistics, Uppsala University \\ Thanks to Aki Vehtari, Aalto University}
\date{}

\begin{document}

\frame{\titlepage
% \thispagestyle{empty}
}

%%%%%%%%%%%%%%%%%%%%%%%%%%%%%%%%%%%%%%%%%%%%%%%%%%%%%%%%%%%%%%%%%%


\section{Introduction}
\frame{\sectionpage}
\begin{frame}
  \frametitle{Binomial: known $\theta$}

  \begin{itemize}
  \item Probability of event 1 in trial is $\theta$
  \item<2-> Probability of event 2 in trial is $1-\theta$
  \item<3-> Probability of several events in independent trials is e.g.\\
    $\theta\theta(1-\theta)\theta(1-\theta)(1-\theta)\ldots$
  \item<4-> If there are $n$ trials and we don't care about the order
    of the events, then the probability that event 1 happens {\color{red}$y$} times
    is
    \begin{align*}
      p({\color{red}y}|\theta,n) = \binom{n}{{\color{red}y}} \theta^{\color{red}y}(1-\theta)^{n-{\color{red}y}}
    \end{align*}
  \end{itemize}

%   \begin{center}
%   \only<2>{\includegraphics[width=9cm]{dbinom1.pdf}}
%   \only<3>{\includegraphics[width=9cm]{dbinom10.pdf}}
%   \only<4>{\includegraphics[width=9cm]{dbinom10b.pdf}}
% \end{center}
\end{frame}

\begin{frame}
  \frametitle{Binomial: known $\theta$}

  \begin{itemize}
  \item {\color{blue}Observation model} (function of {\color{red} $y$}, discrete)
    \begin{align*}
      p({\color{red}y}|\theta,n) = \binom{n}{{\color{red}y}} \theta^{\color{red}y}(1-\theta)^{n-{\color{red}y}}
    \end{align*}
  \end{itemize}

  \begin{center}
    \only<2>{\includegraphics[width=9cm]{figs/dbinom1.pdf}}
  \end{center}
\end{frame}

\begin{frame}
  \frametitle{Binomial: known $\theta$}

  \begin{itemize}
  \item {\color{blue}Observation model} (function of {\color{red} $y$}, discrete)
    \begin{align*}
      p({\color{red}y}|\theta,n) = \binom{n}{{\color{red}y}} \theta^{\color{red}y}(1-\theta)^{n-{\color{red}y}}
    \end{align*}
  \end{itemize}

  \begin{center}
    {\includegraphics[width=9cm]{figs/dbinom10.pdf}\\
      \vspace{-0.6\baselineskip}
\uncover<2>{\hspace{-18mm}\scriptsize    $p({\color{red}y}|n=10,\theta=0.5)$:\, 0.00 0.01 0.04 0.12 0.21 0.25 0.21 0.12 0.04 0.01 0.00}}
\end{center}
\end{frame}

\begin{frame}
  \frametitle{Binomial: known $\theta$}

  \begin{itemize}
  \item {\color{blue}Observation model} (function of {\color{red} $y$}, discrete)
    \begin{align*}
      p({\color{red}y}|\theta,n) = \binom{n}{{\color{red}y}} \theta^{\color{red}y}(1-\theta)^{n-{\color{red}y}}
    \end{align*}
  \end{itemize}

  \begin{center}
  \only<1>{\includegraphics[width=9cm]{figs/dbinom10b.pdf}\\
      \vspace{-0.6\baselineskip}
{\hspace{-18mm}\scriptsize    $p({\color{red}y}|n=10,\theta=0.9)$:\, 0.00 0.00 0.00 0.00 0.00 0.00 0.01 0.06 0.19 0.39 0.35}}
    \only<2>{\includegraphics[width=9cm]{figs/dbinom10.pdf}\\
      \vspace{-0.6\baselineskip}
{\hspace{-22mm}\scriptsize    $p({\color{red}y}=6|n=10,\theta=0.5)$:\, 0.00 0.01 0.04 0.12 0.21 0.25 \textbf{0.21} 0.12 0.04 0.01 0.00}}
  \only<3>{\includegraphics[width=9cm]{figs/dbinom10b.pdf}\\
      \vspace{-0.6\baselineskip}
{\hspace{-22mm}\scriptsize    $p({\color{red}y}=6|n=10,\theta=0.9)$:\, 0.00 0.00 0.00 0.00 0.00 0.00 \textbf{0.01} 0.06 0.19 0.39 0.35}}
\end{center}
\end{frame}

\begin{frame}
  \frametitle{Binomial: unknown $\theta$}

  \begin{itemize}
  \item Posterior with Bayes rule (function of $\theta$, continuous)
    \begin{equation*}
      p(\theta|y,n,M)=\frac{p(y|\theta,n,M)p(\theta|n,M)}{p(y|n,M)}
    \end{equation*}
    \pause
    where $p(y|n,M)=\int p(y|\theta,n,M)p(\theta|n,M) d\theta$
  \item<3-> Start with uniform prior
    \begin{align*}
      p(\theta|n,M)=p(\theta|M)=1,\, \text{when}\,\, 0\leq\theta\leq 1
    \end{align*}
  \item<4-> Then
    \begin{align*}
      p(\theta|y,n,M)&=\frac{p(y|\theta,n,M)}{p(y|n,M)}
      =\frac{\binom{n}{y} \theta^y(1-\theta)^{n-y}}{\int_0^1
        \binom{n}{y} \theta^y(1-\theta)^{n-y} d\theta} \\
        &=\frac{1}{Z}\theta^y(1-\theta)^{n-y}
    \end{align*}
  \end{itemize}

\end{frame}

\begin{frame}
  \frametitle{Binomial: unknown $\theta$}

  \begin{itemize}
  \item Normalization term $Z$ (constant given $y$)
    \begin{equation*}
      Z= \int_0^1 \theta^y(1-\theta)^{n-y} d\theta = \frac{\Gamma(y+1)\Gamma(n-y+1)}{\Gamma(n+2)}
    \end{equation*}
  \item Normalisation term has \emph{Beta} function form
    \begin{itemize}
    \item when integrated over $(0,1)$
      the result can presented with Gamma functions
    \item with integers  $\Gamma(n)=(n-1)!$
    \item for large integers even this is challenging and usually
      $\log \Gamma(\cdot)$ is computed instead of $\Gamma(\cdot)$
    \end{itemize}
  \end{itemize}

\end{frame}

\begin{frame}
  \frametitle{Binomial: unknown $\theta$}

  \begin{itemize}
  \item Posterior is
    \begin{align*}
      p(\theta|y,n,M) = \frac{\Gamma(n+2)}{\Gamma(y+1)\Gamma(n-y+1)}\theta^y(1-\theta)^{n-y},
    \end{align*}
    \only<2>{
    which is called Beta distribution
    \begin{align*}
      \theta|y,n \sim \text{Beta}(y+1,n-y+1)
    \end{align*}}
  \end{itemize}
  \vspace{0.5\baselineskip}
  \begin{center}
    \only<2>{\includegraphics[width=9cm]{figs/dbbeta10c.pdf}}
  \end{center}
\end{frame}

\begin{frame}
  \frametitle{Binomial: computation}

  \begin{itemize}
  \item R
    \begin{itemize}
    \item density {\tt dbeta}
    \item CDF {\tt pbeta}
    \item quantile {\tt qbeta}
    \item random number {\tt rbeta}
    \end{itemize}
  \item Python
    \begin{itemize}
    \item {\tt from scipy.stats import beta}
    \item density {\tt beta.pdf}
    \item CDF {\tt beta.cdf}
    \item prctile {\tt beta.ppf}
    \item random number {\tt beta.rvs}
    \end{itemize}
  \end{itemize}

\end{frame}

\begin{frame}
  \frametitle{Binomial: computation*}

  \begin{itemize}
  \item Beta CDF not trivial to compute
  \item For example, {\tt pbeta} in R uses a continued fraction with
    weighting factors and asymptotic expansion
  \item Laplace developed normal approximation (Laplace
    approximation), because he didn't know how to compute Beta CDF
  \end{itemize}
  % Lagandre, Gamma function

\end{frame}
% Beta distribution named by C. Gini 1911

\begin{frame}
  \frametitle{Placenta previa}

  \begin{itemize}
  \item Probability of a girl birth given placenta previa (BDA3 p. 37)
    \begin{itemize}
    \item 437 girls and 543 boys have been observed
    \item is the ratio 0.445 different from the population average 0.485?
    \end{itemize}
  \end{itemize}
  \pause
  \includegraphics[width=9cm]{figs/demo2_1.pdf}
\end{frame}

%%%% predictive distribution

\begin{frame}
  \frametitle{Predictive distribution -- Effect of integration}

  \begin{itemize}
  \item Predictive distribution for new $\tilde{y}$ (discrete)
    \begin{align*}
      \uncover<2->{p(\tilde{y}=1|y,n,M)} & \uncover<2->{= \int_0^1} p(\tilde{y}=1|\theta,y,n,M) \uncover<2->{p(\theta|y,n,M)d\theta}\\
      &\uncover<3->{= \int_0^1 \theta p(\theta|y,n,M)d\theta}\\
      &\uncover<4->{= \E[\theta|y]}
    \end{align*}
    \vskip -4mm
  \item<5-> With uniform prior
    \begin{align*}
      \uncover<4->{\E[\theta|y] = \frac{y+1}{n+2}}
    \end{align*}
  \item<6-> Extreme cases
    \begin{align*}
      p(\tilde{y}=1|y=0,n,M) &= \frac{1}{n+2} \\
      p(\tilde{y}=1|y=n,n,M) &= \frac{n+1}{n+2}
    \end{align*}
    \vskip -2mm

  \item<6-> cf. maximum likelihood

  \end{itemize}
\end{frame}

\begin{frame}
  \frametitle{Benefits of integration}

  Example: $n=10, y=10$
  \begin{center}
  \includegraphics[width=10cm]{figs/dbbeta10.pdf}
  \end{center}

\end{frame}

\begin{frame}
  \frametitle{Predictive distribution}

  \begin{itemize}
  \item {\color{blue} Prior predictive} distribution for new $\tilde{y}$ (discrete)
    \begin{align*}
      p(\tilde{y}=1|M) &= \int_0^1 p(\tilde{y}=1|\theta,y,n,M){\color{blue}p(\theta|M)}d\theta
    \end{align*}
  \item {\color{red} Posterior predictive} distribution for new $\tilde{y}$ (discrete)
    \begin{align*}
      p(\tilde{y}=1|y,n,M) &= \int_0^1 p(\tilde{y}=1|\theta,y,n,M){\color{red}p(\theta|y,n,M)}d\theta
    \end{align*}
  \end{itemize}
\end{frame}

\begin{frame}
  \frametitle{Justification for uniform prior}

  \begin{itemize}
  \item $p(\theta|M)=1$ if
    \begin{itemize}
    \item[1)] we want the prior predictive distribution to be uniform
      \begin{equation*}
        p(y|n,M) = \frac{1}{n+1}, \quad y=0,\ldots,n
      \end{equation*}
      \begin{itemize}
      \item nice justification as it is based on observables $y$ and $n$
      \end{itemize}
     \item<2->[2)] we think all values of $\theta$ are equally likely
      %   Laplace "principle of insufficient reason"
    \end{itemize}
  \end{itemize}

\end{frame}

%%%priors

\begin{frame}
  \frametitle{Priors}

  \begin{itemize}
  \item Conjugate prior (BDA3 p. 35)
  \item Noninformative prior (BDA3 p. 51)
  \item Proper and improper prior (BDA3 p. 52)
  \item Weakly informative prior (BDA3 p. 55)
  \item Informative prior (BDA3 p. 55)
  \item Prior sensitivity (BDA3 p. 38)
  \end{itemize}

\end{frame}

\begin{frame}

  \frametitle{Conjugate prior}

  \begin{itemize}
  \item Prior and posterior have the same form
    \begin{itemize}
    \item only for exponential family distributions (plus for
      some irregular cases)
    \end{itemize}
  \item Used to be important for computational reasons, and still
    sometimes used for special models to allow partial analytic
    marginalization (Ch 3)
    \begin{itemize}
    \item with dynamic Hamiltonian Monte Carlo used e.g. in Stan no any
      computational benefit
    \end{itemize}
  \end{itemize}

\end{frame}

\begin{frame}

  \frametitle{Beta prior for Binomial model}

  \begin{itemize}
  \item Prior \vskip -1.5\baselineskip
    \begin{align*}
      \text{Beta}(\theta|\alpha,\beta) \propto \theta^{\alpha-1}
      (1-\theta)^{\beta-1}
    \end{align*}
  \item Posterior
    \vskip -1.5\baselineskip
    \begin{align*}
      p(\theta|y,n,M) & \propto \theta^y(1-\theta)^{n-y}
      \theta^{\alpha-1} (1-\theta)^{\beta-1}\\
      &\uncover<2->{ \propto
        \theta^{y+\alpha-1} (1-\theta)^{n-y+\beta-1}}\\
      \uncover<3->{\text{after normalization}} &\\
      \uncover<3->{p(\theta|y,n,M)}
      &\uncover<3->{ = \text{Beta}(\theta|\alpha+y,\beta+n-y)}
    \end{align*}
    \vskip -2mm
  \item<4-> $(\alpha-1)$ and $(\beta-1)$ can considered to be number of prior observations
  \item<4-> Uniform prior when $\alpha=1$ and $\beta=1$
  \end{itemize}
\end{frame}


\begin{frame}
  \frametitle{Placenta previa}

  \begin{itemize}
  \item Beta prior centered on population average 0.485
  \end{itemize}
  \includegraphics[width=11cm]{figs/demo2_2.pdf}
\end{frame}
% We next consider an informative prior.  As discussed in Section \ref{beauty1}, the percentage of girl births is remarkably stable at about 48.8\% girls, rarely varying by more than 0.5\% from this rate.

\begin{frame}
  \frametitle{Benefits of integration and prior}

  \vspace{-0.5\baselineskip}
  Example: $n=10, y=10$ - uniform vs Beta(2,2) prior
  \begin{center}
  \includegraphics[width=7.8cm]{figs/dbbeta10a.pdf}\\
  \includegraphics[width=7.8cm]{figs/dbbeta10b.pdf}
  \end{center}

\end{frame}

\begin{frame}
  \frametitle{Beta prior for Binomial model}

  \begin{itemize}
  \item Posterior
    \vskip -1.5\baselineskip
    \begin{align*}
      p(\theta|y,n,M) = \text{Beta}(\theta|\alpha+y,\beta+n-y)
    \end{align*}
    \item Posterior mean
    \vskip -1\baselineskip
    \begin{align*}
      \E[\theta|y] = \frac{\alpha+y}{\alpha+\beta+n}
    \end{align*}
    \begin{itemize}
    \item combination prior and likelihood information
    \item when $n\rightarrow\infty$, $\text{E}[\theta|y]\rightarrow y/n$
    \end{itemize}
    \pause
  \item  Posterior variance
    \vskip -1\baselineskip
    \begin{align*}
      \var[\theta|y]=\frac{\E[\theta|y](1-\E[\theta|y])}{\alpha+\beta+n+1}
    \end{align*}
    \begin{itemize}
    \item decreases when $n$ increases
    \item when $n\rightarrow\infty$, $\var[\theta|y]\rightarrow 0$
    \end{itemize}
  \end{itemize}

\end{frame}

\begin{frame}

  \frametitle{Noninformative prior, proper and improper prior}

  \begin{itemize}
  \item Vague, flat, diffuse of noninformative
    \begin{itemize}
    \item try to ``to let the data speak for themselves''
    \item flat is not non-informative
    \item flat can be stupid
    \item making prior flat somewhere can make it non-flat somewhere
      else
    \end{itemize}
  \item Proper prior has $\int p(\theta) = 1$
  \item Improper prior density doesn't have a finite integral
    \begin{itemize}
    \item the posterior can still sometimes be proper
    \end{itemize}
  \end{itemize}

\end{frame}



%%%%%%%%%%%%%%%%%%%%%%%%%%%%%%%%%%%%%%%%%%%%%%%%%%%%%%%%%%%%%%%%%%


\end{document}
