%\documentclass[10pt,handout]{beamer}
\documentclass[10pt]{beamer}
\usepackage{babel} % Anpassa efter svenska. Ger svensk logga.
\usepackage[utf8]{inputenc} % Anpassa efter linux
\usepackage{graphicx}
\usepackage{hyperref}
\usepackage{listings}
\input{../common/lststan} % Stan listing


\hypersetup{
    colorlinks=true,
    linkcolor=blue,
    filecolor=magenta,
    urlcolor=cyan,
}
\usepackage{../common/beamerthemeUppsala}
%\usetheme{Uppsala}
%\usecolortheme{UU} % Anpassa efter UU:s frger och logga
%\hypersetup{pdfpagemode=FullScreen} % Adobe Reader ska ppna fullskrm
\setbeamertemplate{itemize items}[circle]

% \usepackage{beamerthemesplit}
\usepackage{amsmath}
% \usepackage{amssymb}
% \usepackage{graphics}
% \usepackage{graphicx}
% \usepackage{epsfig}
% \usepackage[latin1]{inputenc}
 \usepackage{color}
% \usepackage{fancybox}
% \usepackage{psfrag}
% \usepackage[english]{babel}
 \setbeamertemplate{footline}{\hfill\insertframenumber/\inserttotalframenumber}

% Input new commands
\input{../common/commands.tex}

%%%%%%%%%%%%%%%%%%%%%%%%%%%%%%%%%%%%%%%%%%%%%%%%%%%%%%%%%%%%%%%%%%

\setlength{\parskip}{3mm}
\title[]{{\color{black}Bayesian Statistics and Data Analysis \\ Lecture 6}}
\author[]{M{\aa}ns Magnusson \\ Department of Statistics, Uppsala University \\ Thanks to Aki Vehtari, Aalto University}
\date{}

\begin{document}

\frame{\titlepage
% \thispagestyle{empty}
}

%%%%%%%%%%%%%%%%%%%%%%%%%%%%%%%%%%%%%%%%%%%%%%%%%%%%%%%%%%%%%%%%%%


\section{MCMC recap}
%\frame{\sectionpage}

\begin{frame}
\frametitle{Recap: MCMC, Gibbs and Metropolis}

\begin{itemize}
\item Markov Chain Monte Carlo
\begin{itemize}
\item A \uured{transition distribution} $T(\theta_0 \rightarrow \theta_1)$ with a unique \uured{stationary distribution}
\item Target: setup $T$ so that $p(\theta|y)$ is the stationary distribution\pause
\item [+] generic \pause
\item [-] generates dependent draws (inefficiencies/low $S_\text{eff}$)
\item [-] need to assess convergence to $p(\theta|y)$
\end{itemize}
\pause
\item Gibbs sampling
\begin{itemize}
\item Conditional (or block) sampling of $\theta$
\[
\theta_j \sim p(\theta_j|\theta_{-j}, y)
\]
\pause
\item[+] Often easy to construct \pause
\item[-] Inefficient if posterior has correlated parameters
\end{itemize}
\pause
\item Metropolis(-Hastings) sampling
\begin{itemize}
\item Joint (or block) sampling of $\theta$
\item Proposal distribution $J$ (i.e. $T$)\pause
\item[+] better for correlated posteriors\pause
\item[-] scale need to be tuned for efficient sampling
\item[-] hard to propose in high dimensions (many small steps or many rejections)
\end{itemize}
\end{itemize}

\end{frame}

% TODO: Add demo Gibbs + Metropolis-Hastings

\section{Hamiltonian Monte Carlo}
\frame{\sectionpage}

% Motivation: Why HMC?
\begin{frame}
\frametitle{Why Hamiltonian Monte Carlo?}

\begin{itemize}
\item Want to build an \uured{efficient} Markov Chain
\begin{itemize}
\item We want to sample jointly all $\theta$
\item We know the \uured{unnormalized} posterior $q(\theta|y)=Z \cdot p(\theta|y)$, were $Z$ is the normalization constant.
\item Can we use this to create a good \uured{proposal} distribution $J$?
\pause
\item \uured{Hamiltonian Monte Carlo}!
\end{itemize}
\end{itemize}

\end{frame}


\begin{frame}
\frametitle{What is Hamiltonian Monte Carlo?}

\begin{itemize}
\item Add momentum variables to our posterior (\uured{canonical} distribution)
\[
 p(\psi,\theta | y) = p(\psi | \theta , y) \cdot p(\theta | y)\,,
\]
in practice we let $p(\psi | \theta , y) = p(\psi)$
\pause
\item Idea from Physics (Mechanics):
\begin{itemize}
\item $\theta$: position
\item $\psi$: momentum
\end{itemize}
\pause
\item Define the Hamiltonian as
\begin{align}
H(\psi, \theta)  & = - \log(p(\psi)) - \log(p(\theta | y))\\
 & = K(\psi) + V(\theta)\,,
\end{align}
where $K(\psi)$ is the \uured{kinetic} energy and $V(\theta)$ is the \uured{potential} energy
\end{itemize}
\end{frame}

\begin{frame}
\frametitle{What is Hamiltonian Monte Carlo?}

\begin{itemize}
\item Hamiltonian Dynamics (\uured{preserve energy})
\begin{align}
\frac{d\theta}{dt} &=  \frac{\partial H}{\partial \psi} = \frac{\partial K}{\partial \psi}\\
\frac{d\psi}{dt} &=  - \frac{\partial H}{\partial \theta} = \frac{\partial V}{\partial \theta}
\end{align}
\item Let $V(\theta) = - \log(q(\theta|y))= - \log p(\theta) - \log p(y | \theta)$
\item Let $\psi \sim N(0, M)$ where $M$ is the \uured{mass matrix}
\item Hence, $K(\psi) = - \log p(\psi) \propto 0.5 \psi^T M^{-1} \psi + C$
\pause
\item We need to choose $M$ in a smart way.
\begin{enumerate}
\item Ideally, $M^{-1}=Cov(\theta|y)$
\item In practice, $M^{-1}=V(\theta|y)$
\end{enumerate}
% TODO: readup on the mass matrix and why inverse mass matrix is Cov
\end{itemize}
\end{frame}


\begin{frame}

\frametitle{The leapfrog integrator}

\begin{itemize}
\item We want to simulate Hamiltonian dynamics
\begin{align}
\frac{d\theta}{dt} &=  M^{-1} \psi\\
\frac{d\psi}{dt} &=  \frac{\partial \log q(\theta|y)}{\partial \theta}
\end{align}
\item A \uured{discrete} approximation: \uured{the leapfrog integrator}
\pause
\item We take $L$ leapfrog steps with step size $\epsilon$ as
\begin{align}
\psi \leftarrow &\psi + \frac{1}{2}\epsilon \frac{d\log q(\theta|y)}{d\theta}\\
\theta \leftarrow& \theta + \epsilon M^{-1}\psi\\
\psi \leftarrow &\psi + \frac{1}{2}\epsilon \frac{d\log q(\theta|y)}{d\theta}
\end{align}
\pause
\item Discretization introduce a error depending on $\epsilon$ (not $L \epsilon$)
\end{itemize}
\end{frame}

\begin{frame}

\frametitle{Hamiltonian Monte Carlo Algorithm}

\begin{enumerate}
\item Sample momentum
\[
\psi_0 \sim N(0,M)
\]
\pause
\item Simulate values $(\theta^\star,\psi^\star)$ using the leapfrog integrator $L$ steps with stepsize $\epsilon$, starting from $(\theta_{t-1},\psi_0)$
\pause
\item Accept the proposed values $(\theta^\star,\psi^\star)$ with probability
\[
r = \min \left( 1, \frac{q(\theta^\star|y)}{q(\theta_{t-1}|y)}\frac{p(\psi^\star)}{p(\psi_0)} \right)
\]
\end{enumerate}
\end{frame}


\begin{frame}

\frametitle{Hamiltonian Monte Carlo}

%  \vspace{-0.5\baselineskip}
  \begin{itemize}
  \item Bivariate Normal HMC example
%  \item Uses gradient of log density for more efficient sampling
%  \item Augments parameter space with momentum variables
  \end{itemize}

  \vspace{-0.5\baselineskip}
  \only<1>{\phantom{\includegraphics[width=7cm]{figs/hmcdemo01.pdf}}}
  \only<2>{\includegraphics[width=7cm]{figs/hmcdemo01.pdf}}
  \only<3>{\includegraphics[width=7cm]{figs/hmcdemo02.pdf}}
  \only<4>{\includegraphics[width=7cm]{figs/hmcdemo03.pdf}}
  \only<5>{\includegraphics[width=7cm]{figs/hmcdemo04.pdf}}
  \only<6>{\includegraphics[width=7cm]{figs/hmcdemo05.pdf}}
  \only<7>{\includegraphics[width=7cm]{figs/hmcdemo06.pdf}}
  \only<8>{\includegraphics[width=7cm]{figs/hmcdemo07.pdf}}
  \only<9>{\includegraphics[width=7cm]{figs/hmcdemo08.pdf}}
  \only<10>{\includegraphics[width=7cm]{figs/hmcdemo09.pdf}}
  \only<11>{\includegraphics[width=7cm]{figs/hmcdemo10.pdf}}
  \only<12>{\hspace{-5mm}\includegraphics[width=9cm]{figs/hmc1trace.pdf}}
  \only<13>{\hspace{-5mm}\includegraphics[width=9cm]{figs/hmc1acf.pdf}}
%  \only<14>{\hspace{-5mm}\includegraphics[width=9cm]{figs/hmc1mcerr.pdf}}

\end{frame}


\begin{frame}
\frametitle{Hamiltonian Monte Carlo Summary}

\begin{itemize}
\item Parameters:
\begin{itemize}
\item[$\epsilon$] step size \pause
\item[$L$] leapfrog steps\pause
\item[$M$] mass matrix\pause
\end{itemize}
\item[+] Can be very efficient ($S_\text{eff}$)
\item[+] Additional diagnostics
\pause
\item[-] Can be difficult to tune (U-turns)
\item[-] Bounded parameters needs handling
\item[-] Cannot handle discrete parameters (yet)
\item[-] Ideally, we should adapt $\epsilon L$
\item[-] Costly to run each iteration ($L$ log density gradient evaluations)
\pause
\end{itemize}

\centering
\href{https://chi-feng.github.io/mcmc-demo/app.html?algorithm=HamiltonianMC&target=standard}{demo}

\end{frame}


% TODO: Work through a mathematical example

\section{Dynamic HMC and NUTS}
\frame{\sectionpage}

\begin{frame}
\frametitle{Dynamic HMC}

  \begin{itemize}
  \item \uured{Goal:} Simplify/adapt the tuning of HMC
  \item \uured{Dynamic} HMC refers to \uured{dynamic} trajectory length of the leapfrog integrator (i.e. $L$ is chosen on the fly) \pause
  \item The NUTS/dynamic algorithm:
  \begin{enumerate}
    \item Grow a binary tree of leapfrog steps $L$
    \pause
    \item Grow (randomly) in two directions \\
    (to keep reversibility/detailed balance of Markov chain)
    \pause
    \item Stop to grow tree when encounter a U-turn
    \[
    (\theta_L - \theta_{start}) \cdot \psi_L < 0
    \]
    \item Sample one of the accepted steps at the trajectory\\
    (higher probability further away)\\
  \end{enumerate}

  \pause
  \item Dynamic simulation is discretized
  \begin{itemize}
    \item small $\epsilon$ gives accurate simulation, but requires more log density evaluations
    \item large $\epsilon$ reduces computation, but increases
      simulation error
    \end{itemize}
\end{itemize}

\end{frame}

\begin{frame}
\frametitle{Dynamic Hamiltonian Monte Carlo Summary}

\begin{itemize}
\item Parameters:
\begin{itemize}
\item[$\epsilon$] step size \pause
\item[$M$] mass matrix\pause
\end{itemize}
\item[+] Can be very efficient ($S_\text{eff}$)
\item[+] Additional diagnostics
\pause
\item[-] Bounded parameters needs handling
\item[-] Cannot handle discrete parameters (yet)
\item[-] Costly to run each iteration ($L$ log density gradient evaluations)
\pause
\end{itemize}

\centering
\href{https://chi-feng.github.io/mcmc-demo/app.html?algorithm=EfficientNUTS&target=standard}{demo}

\end{frame}

\begin{frame}

\frametitle{HMC / NUTS}

  \vspace{-.5\baselineskip}
  \includegraphics[width=\textwidth,clip]{figs/N250.pdf}\\
  Source: Jonah Gabry

\end{frame}

\section{HMC diagnostics}
\frame{\sectionpage}

\begin{frame}

\frametitle{Max tree depth}

  \begin{itemize}
  \item Dynamic HMC specific diagnostic
  \item The sampler wanted to keep integrating longer, but was forced to stop.
  \begin{itemize}
  \item The trajectory wasn’t long enough to fully explore the typical set.
  \item This often signals:
  \begin{itemize}
  \item Very strong correlations or curvature in the posterior (forcing small step size)
  \item A need for reparameterization.
  \end{itemize}
  \end{itemize}
  \item Leads to higher autocorrelations and lower ESS ($n_{\rm eff}$)
  \item Different parameterizations can help/matter
  \end{itemize}
\end{frame}

\begin{frame}
\frametitle{Divergent transitions}

% What
\begin{itemize}
  \item HMC specific diagnostic
  \pause
  \item The Hamiltonian $H(\theta, \psi) $ should remain roughly constant throughout the trajectory:
  \[
  H(\theta_{start}, \psi_{start}) \approx H(\theta_{end}, \psi_{end})
  \]
  \item A large difference between the initial and final Hamiltonian indicates a divergent transition:
  \[
  H(\theta_{start}, \psi_{start}) - H(\theta_{end}, \psi_{end}) >> 0
  \]
  \item indicates that Hamiltonian dynamic simulation has problems with unexpected fast changes in log-density
  \item possibility of biased estimates
\end{itemize}
%\end{frame}

% Why


\end{frame}

\begin{frame}
\frametitle{Divergent transitions}

% Why
\begin{itemize}
  \item Why do we get divergent transitions?
  \begin{enumerate}
    \item Step size too large
    \begin{itemize}
      \item Do not approximate the trajectory
    \end{itemize}
    \pause
    \item Highly curved posterior geometry
    \begin{itemize}
      \item Small changes in position can result in very large changes in the gradient (momentum)
      \item Unstable trajectory
      \item Common in hiearhical models
    \end{itemize}
    \pause
    \item Tight correlations between parameters
    \begin{itemize}
      \item Small changes in position can result in very large changes in the gradient (momentum)
      \item small movements in one parameter may cause disproportionately large movements in another
    \end{itemize}
    \end{enumerate}
\end{itemize}

\end{frame}

\begin{frame}
\frametitle{Divergent transitions}

% Solutions
\begin{itemize}
  \item Solutions:
  \begin{enumerate}
    \item Decrease step size/Increase number of leapfrog steps
    \pause
    \item Reparameterization
    \begin{itemize}
      \item Reparameterizing the model to reduce correlations between parameters or to make the posterior geometry more regular can help, e.g. in hiearhical models
    \end{itemize}
    \pause
    \item Use a better Metric (Mass Matrix)
    \pause
    \item Use other samplers that uses the Hessian, e.g. Riemannian Manifold HMC (but it has its own drawbacks)
    \end{enumerate}
\end{itemize}

\end{frame}


\begin{frame}
\frametitle{Divergences (Betancourt, 2017)}

%     \only<1>{\phantom{\includegraphics[width=8cm]{figs/unnamed-chunk-6-1.png}}}
%     \only<2>{\includegraphics[width=8cm]{figs/unnamed-chunk-6-1.png}}
%     \only<3>{\includegraphics[width=8cm]{figs/unnamed-chunk-7-1.png}}
     \only<1>{\includegraphics[width=8cm]{figs/unnamed-chunk-9-1.png}}
%     \only<2>{\includegraphics[width=8cm]{figs/unnamed-chunk-12-1.png}}
%     \only<3>{\includegraphics[width=8cm]{figs/unnamed-chunk-13-1.png}}
     \only<2>{\includegraphics[width=8cm]{figs/unnamed-chunk-15-1.png}}
     \only<3>{\includegraphics[width=8cm]{figs/unnamed-chunk-21-1.png}}
     \only<4>{\includegraphics[width=8cm]{figs/unnamed-chunk-31-1.png}}
\end{frame}


\begin{frame}

\frametitle{Problematic distributions}

  \begin{itemize}
  \item<1-> Nonlinear dependencies
    \begin{itemize}
    \item optimal proposal depends on location
    \end{itemize}
    \begin{center}
      \href{https://chi-feng.github.io/mcmc-demo/app.html?algorithm=RandomWalkMH&target=banana}{demo}
    \end{center}
  \item<2-> Funnels
    \begin{itemize}
    \item optimal proposal depends on location
    \end{itemize}
    \begin{center}
      \href{https://chi-feng.github.io/mcmc-demo/app.html?algorithm=RandomWalkMH&target=funnel}{demo}
    \end{center}
  \item<3-> Multimodal
    \begin{itemize}
    \item difficult to move from one mode to another
    \end{itemize}
        \begin{center}
      \href{https://chi-feng.github.io/mcmc-demo/app.html?algorithm=RandomWalkMH&target=multimodal}{demo}
    \end{center}
  \item<4-> Non-identifiable models
    \begin{itemize}
    \item set of connected points is the mode
    \end{itemize}
        \begin{center}
      \href{https://chi-feng.github.io/mcmc-demo/app.html?algorithm=RandomWalkMH&target=donut}{demo}
    \end{center}
  \item<5-> Long-tailed with non-finite variance and mean
    \begin{itemize}
    \item central limit theorem for expectations does not hold
    \end{itemize}

  \end{itemize}

\end{frame}


%\begin{frame}

%\frametitle{Extra (optional) material for HMC}

%  \begin{itemize}
%  \item Michael Betancourt (2018).  A Conceptual Introduction to
%    Hamiltonian Monte Carlo. \url{https://arxiv.org/abs/1701.02434}
%  \item Michael Betancourt (2017).  Diagnosing Biased Inference with Divergences. \url{https://mc-stan.org/users/documentation/case-studies/divergences_and_bias.html}
%  \end{itemize}
%\end{frame}


\section{Probabilistic Programming}
\frame{\sectionpage}


\begin{frame}{The Box process: Probabilistic modeling}

\begin{figure}
    \centering
    \includegraphics[width=1\textwidth]{figs/Boxs_loop.png}
    \caption{The Box approach (Box, 1976, Blei, 2014)}
\end{figure}
\end{frame}


\begin{frame}
\frametitle{Probabilistic programming languages}

  \begin{itemize}
  \item Wikipedia ``A probabilistic programming language (PPL) is a
    programming language designed to describe probabilistic models
    and then perform inference in those models''
    \pause
  \item To make probabilistic programming useful
    \begin{itemize}
    \item easy workflow to build and revise models
    \item inference has to be as automatic as possible
    \item diagnostics for telling if the automatic inference doesn't work
    \end{itemize}
  \end{itemize}
\end{frame}

\begin{frame}

\frametitle{Probabilistic programming}

  \begin{itemize}
  \item Enables agile (incremental) workflow for developing probabilistic models
    \begin{itemize}
    \item language
    \item automated inference
    \item diagnostics
    \end{itemize}
  \item Many frameworks
    Stan, PyMC3, Pyro (Uber), Edward (Google), Birch (Uppsala), ...
  \end{itemize}

\end{frame}

\section{Stan}
\frame{\sectionpage}

\begin{frame}

  \frametitle{Stan - probabilistic programming framework}

   \begin{itemize}
   \item Language, inference engine, user interfaces, documentation,
     case studies, diagnostics, packages, ...
     \begin{itemize}
     \item autodiff to compute gradients of the log density
     \end{itemize}
   \item<2-> More than ten thousand users in social, biological, and
     physical sciences, medicine, engineering, and business

   \item<3-> Several full time developers, 40+ developers, more than 100 contributors
   \item<4-> R, Python, Julia, Scala, Stata, Matlab, command line interfaces
    \item<4-> More than 100 R packages using Stan
   \end{itemize}
  \vfill
  \begin{center}
    \includegraphics[width=1.5cm]{figs/stan_logo_wide.png}\\
    \url{mc-stan.org}
  \end{center}
\end{frame}

\begin{frame}

\frametitle{Stan}

  \begin{itemize}
  \item Stanislaw Ulam (1909-1984)
    \begin{itemize}
    \item Monte Carlo method
    \item H-Bomb
    \end{itemize}
  \end{itemize}

\end{frame}


\begin{frame}
\frametitle{Adaptive dynamic HMC in Stan}

  \begin{itemize}
  \item Dynamic HMC using growing tree to increase simulation
    trajectory until no-U-turn criterion stopping
    \begin{itemize}
    \item max treedepth to keep computation in control
    \item<2-> pick a draw along the trajectory with probabilities adjusted
      to take into account the error in the discretized dynamic
      simulation
    \item<3-> give bigger weight for tree parts further away to increase
      probability of jumping further away
    \end{itemize}
  \item<4-> Mass matrix and step size adaptation in Stan
    \begin{itemize}
    \item<4-> mass matrix refers to having different scaling for different
      parameters and optionally also rotation to reduce correlations
    \item<5-> mass matrix and step size adjustment and are estimated
      during initial adaptation phase
    \item<6-> step size is adjusted to be as big as possible while keeping
      discretization error in control (adapt\_delta)
    \end{itemize}
  \item<7-> After adaptation the algorithm parameters are fixed
  \item<8-> After warmup store iterations for inference
  \item<9-> See more details in Stan reference manual
\end{itemize}

\end{frame}

\begin{frame}[fragile]

\frametitle{Binomial model - Stan code}
  {\small\color{gray}
{\only<1>{\color{black}}
  \begin{lstlisting}[language=Stan,basicstyle=\ttfamily]
data {
  int<lower=0> N;     // number of experiments
  int<lower=0,upper=N> y; // number of successes
}
\end{lstlisting}}
  {\only<2>{\color{black}}
\begin{lstlisting}[language=Stan]
parameters {
  real<lower=0,upper=1> theta; // parameter of the binomial
}
\end{lstlisting}}
{\only<3>{\color{black}}
\begin{lstlisting}[language=Stan]
model {
  theta ~ beta(1,1);     //prior
  y ~ binomial(N,theta); // observation model
}
\end{lstlisting}
}}
\end{frame}

\begin{frame}[fragile]

\frametitle{Binomial model - Stan code}

  {\small
  \begin{lstlisting}[language=Stan]
data {
  int<lower=0> N;     // number of experiments
  int<lower=0,upper=N> y; // number of successes
}
\end{lstlisting}}

  \begin{itemize}
  \item Data type and size are declared
  \item Stan checks that given data matches type and constraints
    \begin{itemize}
    \item<2-> If you are not used to strong typing, this may
      feel annoying, but it will reduce the probability of coding
      errors, which will reduce probability of data analysis errors
    \end{itemize}
  \end{itemize}
\end{frame}


\begin{frame}[fragile]

\frametitle{Binomial model - Stan code}

  {\small
\begin{lstlisting}[language=Stan]
parameters {
  real<lower=0,upper=1> theta;
}
\end{lstlisting}}

  \begin{itemize}
  \item Parameters may have constraints
  \item Stan makes transformation to unconstrained space and samples in unconstrained space
    \begin{itemize}
    \item e.g. log transformation for \texttt{<lower=a>}
    \item e.g. logit transformation for \texttt{<lower=a,upper=b>}
    \end{itemize}
  \item<2-> For these declared transformation Stan automatically takes
    into account the Jacobian of the transformation (see BDA3 p. 21)
  \end{itemize}
\end{frame}

\begin{frame}[fragile]

\frametitle{Binomial model - Stan code}

  {\small
\begin{lstlisting}[language=Stan]
model {
  theta ~ beta(1,1);     // prior
  y ~ binomial(N,theta); // likelihood
}
\end{lstlisting}}

\end{frame}

\begin{frame}[fragile]

\frametitle{Binomial model - Stan code}

  {\small
\begin{lstlisting}[language=Stan]
model {
  theta ~ beta(1,1);     // prior
  y ~ binomial(N,theta); // likelihood
}
\end{lstlisting}}

    \vspace{-0.5\baselineskip}
    \begin{itemize}
    \item $\sim$ is syntactic sugar and this is equivalent to
    \end{itemize}

  {\small
\begin{lstlisting}[language=Stan]
model {
  target += beta_lpdf(theta | 1, 1);
  target += binomial_lpmf(y | N, theta);
}
\end{lstlisting}}

    \vspace{-0.5\baselineskip}
    \begin{itemize}
    \item<2-> {\tt target} is the log posterior density
    \item<3-> {\tt \_lpdf} for continuous, {\tt \_lpmf} for discrete distributions (discrete for the left hand side of {\tt |})
    \item<4-> for Stan sampler there is no difference between prior and likelihood, all that matters is the final {\tt target}
    \item<5-> you can write in Stan language any program to compute the
      log density (Stan language is Turing complete)
    \end{itemize}

\end{frame}

% \begin{frame}[fragile]

% \frametitle{Binomial model - Stan code}

%   {\small
% \begin{lstlisting}[language=Stan]
% model {
%   theta ~ beta(1,1);     //prior
%   y ~ binomial(N,theta); // observation model
% }
% \end{lstlisting}}

%     \begin{itemize}
%     \item $\sim$ is syntactic sugar and this could be also written as
%     \end{itemize}

%   {\small
% \begin{lstlisting}[language=Stan]
% model {
%   target +=  beta_lpdf(theta | 1, 1);
%   target +=  binomial_lpmf(y | N, theta);
% }
% \end{lstlisting}}

%     \begin{itemize}
%     \end{itemize}

% \end{frame}

\begin{frame}

\frametitle{Stan}

  \begin{itemize}
  \item Stan compiles (transplies) the model written in Stan language to C++
    \begin{itemize}
    \item this makes the sampling for complex models and bigger data faster
    \item also makes Stan models easily portable, you can use your own
      favorite interface
    \end{itemize}
  \end{itemize}

\end{frame}

\begin{frame}[fragile]

\frametitle{RStan}

  {\small\color{gray}
    {\only<1>{\color{black}}
      RStan
\begin{lstlisting}[language=R]
library(rstan)
rstan_options(auto_write = TRUE)
options(mc.cores = parallel::detectCores())
\end{lstlisting}
    }
{\only<2>{\color{black}}
\begin{lstlisting}[language=R]
d_bin <- list(N = 10, y = 7)
fit_bin <- stan(file = 'binom.stan', data = d_bin)
\end{lstlisting}
}
}
\end{frame}

\begin{frame}[fragile]

\frametitle{PyStan}

  {\small\color{gray}
{\only<1>{\color{black}}
      PyStan
\begin{lstlisting}
import pystan
import stan_utility
\end{lstlisting}
    }
    {\only<2>{\color{black}}
\begin{lstlisting}
data = dict(N=10, y=8)
model = stan_utility.compile_model('binom.stan')
fit = model.sampling(data=data)
\end{lstlisting}
    }
  }
\end{frame}

\begin{frame}

\frametitle{Stan}

  \begin{itemize}
  \item Compilation (unless previously compiled model available)
  \item Warm-up including adaptation
  \item Sampling
  \item Generated quantities
  \item Save posterior draws
  \item Report divergences, $n_\eff$, $\widehat{R}$
  \end{itemize}

\end{frame}

\begin{frame}[fragile]

\frametitle{Difference between proportions}

\begin{itemize}
  \item An experiment was performed to estimate the effect of
    beta-blockers on mortality of cardiac patients
  \item A group of
    patients were randomly assigned to treatment and control groups:
    \begin{itemize}
    \item out of 674 patients receiving the control, 39 died
    \item out of 680 receiving the treatment, 22 died
    \end{itemize}
  \end{itemize}
\end{frame}

\begin{frame}[fragile]

\frametitle{Difference between proportions}

%  Beta-blockers $N_1 = 674, y_1 = 39, N_2 = 680, y_2 = 22$

  {\small\color{gray}
    {\only<1>{\color{black}}
\begin{lstlisting}[language=Stan]
data {
  int<lower=0> N1;
  int<lower=0> y1;
  int<lower=0> N2;
  int<lower=0> y2;
}
parameters {
  real<lower=0,upper=1> theta1;
  real<lower=0,upper=1> theta2;
}
model {
  theta1 ~ beta(1,1);
  theta2 ~ beta(1,1);
  y1 ~ binomial(N1,theta1);
  y2 ~ binomial(N2,theta2);
}
\end{lstlisting}
    }
    {\only<2>{\color{black}}
\begin{lstlisting}[language=Stan]
generated quantities {
  real oddsratio;
  oddsratio = (theta2/(1-theta2))/(theta1/(1-theta1));
}
\end{lstlisting}
    }
  }
\end{frame}

\begin{frame}[fragile]

\frametitle{Difference between proportions}

%  Beta-blockers $N_1 = 674, y_1 = 39, N_2 = 680, y_2 = 22$

  {\small
\begin{lstlisting}[language=Stan]
generated quantities {
  real oddsratio;
  oddsratio = (theta2/(1-theta2))/(theta1/(1-theta1));
}
\end{lstlisting}
    }

    \begin{itemize}
    \item generated quantities is run after the sampling
    \end{itemize}

\end{frame}

\begin{frame}[fragile]

\frametitle{Difference between proportions}

  {\small
\begin{lstlisting}[language=R]
d_bin2 <- list(N1 = 674, y1 = 39, N2 = 680, y2 = 22)
fit_bin2 <- stan(file = 'binom2.stan', data = d_bin2)
\end{lstlisting}
  }

  {\tiny
\begin{lstlisting}
starting worker pid=10151 on localhost:11783 at 10:03:27.872
starting worker pid=10164 on localhost:11783 at 10:03:28.087
starting worker pid=10176 on localhost:11783 at 10:03:28.295
starting worker pid=10185 on localhost:11783 at 10:03:28.461

SAMPLING FOR MODEL 'binom2' NOW (CHAIN 1).

Gradient evaluation took 6e-06 seconds
1000 transitions using 10 leapfrog steps per transition would take 0.06 seconds.
Adjust your expectations accordingly!


Iteration:    1 / 2000 [  0%]  (Warmup)
Iteration:  200 / 2000 [ 10%]  (Warmup)
...
Iteration: 1000 / 2000 [ 50%]  (Warmup)
Iteration: 1001 / 2000 [ 50%]  (Sampling)
...
Iteration: 2000 / 2000 [100%]  (Sampling)

 Elapsed Time: 0.012908 seconds (Warm-up)
               0.017027 seconds (Sampling)
               0.029935 seconds (Total)


SAMPLING FOR MODEL 'binom2' NOW (CHAIN 2).
...
\end{lstlisting}
  }

\end{frame}

\begin{frame}[fragile]

\frametitle{Difference between proportions}

  {\small
\begin{lstlisting}[language=R]
monitor(fit_bin2, probs = c(0.1, 0.5, 0.9))
\end{lstlisting}
  }

  {\scriptsize
\begin{lstlisting}
Inference for the input samples
(4 chains: each with iter=1000; warmup=0):

            mean se_mean  sd    10%    50%    90% n_eff Rhat
theta1       0.1       0 0.0    0.0    0.1    0.1  3280    1
theta2       0.0       0 0.0    0.0    0.0    0.0  3171    1
oddsratio    0.6       0 0.2    0.4    0.6    0.8  3108    1
lp__      -253.5       0 1.0 -254.8 -253.2 -252.6  1922    1

For each parameter, n_eff is a crude measure of effective sample size,
and Rhat is the potential scale reduction factor on split chains (at
convergence, Rhat=1).
\end{lstlisting}
  }

  \begin{itemize}
  \item<2-> {\tt lp\_\_} is the log density, ie, same as {\tt target}
  \end{itemize}

\end{frame}

\begin{frame}[fragile]

\frametitle{Difference between proportions}

  {\small
\begin{lstlisting}
draws <- as.data.frame(fit_bin2)
mcmc_hist(draws, pars = 'oddsratio') +
  geom_vline(xintercept = 1) +
  scale_x_continuous(breaks = c(seq(0.25,1.5,by=0.25)))
\end{lstlisting}
  }

  \begin{center}
  \includegraphics[width=9cm]{figs/betablockoddsratio.pdf}
\end{center}
\end{frame}


\begin{frame}[fragile]

\frametitle{HMC specific diagnostics}

  {\scriptsize
\begin{lstlisting}
check_treedepth(fit_bin2)
check_div(fit_bin2)

[1] "0 of 4000 iterations saturated the maximum tree depth of 10 (0%)"
[1] "0 of 4000 iterations ended with a divergence (0%)"

get_num_leapfrog_per_iteration(fit_bin2)
\end{lstlisting}
  }

\end{frame}

\begin{frame}[fragile]

\frametitle{Shinystan}

  \begin{itemize}
  \item Graphical user interface for analysing MCMC results
  \end{itemize}

\end{frame}

%\begin{frame}
%
%\frametitle{Kilpisjärvi summer temperature}
%
%  \begin{itemize}
%  \item Temperature at Kilpisjärvi in June, July and August from 1952 to 2013
%  \item Is there change in the temperature?
%  \end{itemize}
%  \begin{center}
%    \includegraphics[width=8cm]{figs/kilpis_data.pdf}
%  \end{center}

%\end{frame}

\begin{frame}[fragile]

\frametitle{Gaussian linear model}
  {\small
  \begin{lstlisting}[language=Stan]
data {
    int<lower=0> N; // number of data points
    vector[N] x; //
    vector[N] y; //
}
parameters {
    real alpha;
    real beta;
    real<lower=0> sigma;
}
transformed parameters {
    vector[N] mu;
    mu <- alpha + beta*x;
}
model {
    y ~ normal(mu, sigma);
}
  \end{lstlisting}
}
\end{frame}

\begin{frame}[fragile]

\frametitle{Gaussian linear model}
  {\small
  \begin{lstlisting}[language=Stan]
data {
    int<lower=0> N; // number of data points
    vector[N] x; //
    vector[N] y; //
}
\end{lstlisting}
  }

  \begin{itemize}
  \item difference between {\tt vector[N] x}\, and\, {\tt real x[N]}
  \end{itemize}
\end{frame}

\begin{frame}[fragile]

\frametitle{Gaussian linear model}
  {\small
  \begin{lstlisting}[language=Stan]
parameters {
    real alpha;
    real beta;
    real<lower=0> sigma;
}
transformed parameters {
    vector[N] mu;
    mu <- alpha + beta*x;
}
\end{lstlisting}
  }
  \begin{itemize}
  \item transformed parameters are deterministic transformations of parameters and data
  \end{itemize}
\end{frame}

\begin{frame}[fragile]

\frametitle{Priors for Gaussian linear model}
  {\small
  \begin{lstlisting}[language=Stan]
data {
    int<lower=0> N; // number of data points
    vector[N] x; //
    vector[N] y; //
    real pmualpha; // prior mean for alpha
    real psalpha;  // prior std for alpha
    real pmubeta;  // prior mean for beta
    real psbeta;   // prior std for beta
}
...
transformed parameters {
    vector[N] mu;
    mu <- alpha + beta*x;
}
model {
    alpha ~ normal(pmualpha,psalpha);
    beta ~ normal(pmubeta,psbeta);
    y ~ normal(mu, sigma);
}
  \end{lstlisting}
}
\end{frame}

\begin{frame}[fragile]

\frametitle{Student-t linear model}
  {\small
  \begin{lstlisting}[language=Stan]
...
parameters {
  real alpha;
  real beta;
  real<lower=0> sigma;
  real<lower=1,upper=80> nu;
}
transformed parameters {
  vector[N] mu;
  mu <- alpha + beta*x;
}
model {
  nu ~ gamma(2,0.1);
  y ~ student_t(nu, mu, sigma);
}
  \end{lstlisting}
}
\end{frame}

%\begin{frame}

%\frametitle{Priors}
%
%  \begin{itemize}
%  \item Prior for temperature increase?
%  \end{itemize}
%
%\end{frame}

%\begin{frame}

%\frametitle{Kilpisjärvi summer temperature}

%  Posterior fit
%
%  \begin{center}
%    \includegraphics[width=8cm]{figs/kilpis_lin_pfit.pdf}
%  \end{center}

%\end{frame}

%\begin{frame}[fragile]

%\frametitle{Kilpisjärvi summer temperature}

%  Posterior draws of alpha and beta

%  \begin{center}
%    \includegraphics[width=8cm]{figs/kilpis_lin_mcmc_scatter.pdf}
%  \end{center}

%\end{frame}

%\begin{frame}[fragile]

%\frametitle{Kilpisjärvi summer temperature}

%  Posterior draws of alpha and beta

%  \begin{center}
%    \includegraphics[width=18cm]{figs/kilpis_lin_mcmc_scatter.pdf}
%  \end{center}

%{\scriptsize
%\begin{lstlisting}
%There were 14 transitions after warmup that exceeded the maximum
%treedepth. Increase max_treedepth above 10. See
%http://mc-stan.org/misc/warnings.html#maximum-treedepth-exceeded
%Examine the pairs() plot to diagnose sampling problems
%\end{lstlisting}
%}

%\end{frame}

\begin{frame}[fragile]

\frametitle{Linear regression model in Stan}

  {\scriptsize
\begin{lstlisting}[language=Stan]
data {
  int<lower=0> N; // number of data points
  vector[N] x; //
  vector[N] y; //
  real xpred; // input location for prediction
}
transformed data {
  vector[N] x_std;
  vector[N] y_std;
  real xpred_std;
  x_std = (x - mean(x)) / sd(x);
  y_std = (y - mean(y)) / sd(y);
  xpred_std = (xpred - mean(x)) / sd(x);
}
\end{lstlisting}
  }
\end{frame}

%XXXX add better posterior draws

\begin{frame}[fragile]

\frametitle{RStanARM}

  \begin{itemize}
  \item RStanARM provides simplified model description with
    pre-compiled models
    \begin{itemize}
    \item no need to wait for compilation
    \item a restricted set of models
    \end{itemize}
  \end{itemize}

Two group Binomial model:
  {\scriptsize
\begin{lstlisting}
d_bin2 <- data.frame(N = c(674, 680), y = c(39,22), grp2 = c(0,1))
fit_bin2 <- stan_glm(y/N ~ grp2, family = binomial(), data = d_bin2,
                    weights = N)
\end{lstlisting}
  }
% \begin{lstlisting}[language=R]
% draws_bin2 <- as.data.frame(fit_bin2) %>%
%   mutate(theta1 = plogis(`(Intercept)`),
%          theta2 = plogis(`(Intercept)` + grp2),
%          oddsratio = (theta2/(1-theta2))/(theta1/(1-theta1)))

% mcmc_hist(draws_bin2, pars='oddsratio')
% \end{lstlisting}
%     }

\end{frame}

\begin{frame}[fragile]

\frametitle{RStanARM}

  \begin{itemize}
  \item RStanARM provides simplified model description with
    pre-compiled models
    \begin{itemize}
    \item no need to wait for compilation
    \item a restricted set of models
    \end{itemize}
  \end{itemize}

Two group Binomial model:
  {\scriptsize
\begin{lstlisting}
d_bin2 <- data.frame(N = c(674, 680), y = c(39,22), grp2 = c(0,1))
fit_bin2 <- stan_glm(y/N ~ grp2, family = binomial(), data = d_bin2,
                    weights = N)
\end{lstlisting}
  }
    Gaussian linear model
  {\scriptsize
\begin{lstlisting}
    fit_lin <- stan_glm(temp ~ year, data = d_lin)
\end{lstlisting}
  }
% \begin{lstlisting}[language=R]
% draws_bin2 <- as.data.frame(fit_bin2) %>%
%   mutate(theta1 = plogis(`(Intercept)`),
%          theta2 = plogis(`(Intercept)` + grp2),
%          oddsratio = (theta2/(1-theta2))/(theta1/(1-theta1)))

% mcmc_hist(draws_bin2, pars='oddsratio')
% \end{lstlisting}
%     }

\end{frame}


\begin{frame}[fragile]

\frametitle{BRMS}

  \begin{itemize}
  \item BRMS provides simplified model description
    \begin{itemize}
    \item a larger set of models than RStanARM, but still restricted
    \item need to wait for the compilation
    \end{itemize}
  \end{itemize}

  {\scriptsize
\begin{lstlisting}[language=R]
fit_bin2 <- brm(y/N ~ grp2, family = binomial(), data = d_bin2,
                    weights = N)


fit_lin_t <- brm(temp ~ year, data = d_lin, family = student())
\end{lstlisting}
    }

\end{frame}

\begin{frame}

\frametitle{Extreme value analysis}

Geomagnetic storms

\includegraphics[width=8cm]{figs/stan_gpareto_geomev.png}

\end{frame}

\begin{frame}[fragile]

\frametitle{Extreme value analysis}
  {\small
  \begin{lstlisting}[language=Stan]
data {
  int<lower=0> N;
  vector<lower=0>[N] y;
  int<lower=0> Nt;
  vector<lower=0>[Nt] yt;
}
transformed data {
  real ymax;
  ymax <- max(y);
}
parameters {
  real<lower=0> sigma;
  real<lower=-sigma/ymax> k;
}
model {
  y ~ gpareto(k, sigma);
}
generated quantities {
  vector[Nt] predccdf;
  predccdf<-gpareto_ccdf(yt,k,sigma);
}
  \end{lstlisting}
}
\end{frame}

\begin{frame}[fragile]

\frametitle{Functions}
  {\footnotesize
  \begin{lstlisting}[language=Stan]
functions {
  real gpareto_lpdf(vector y, real k, real sigma) {
    // generalised Pareto log pdf with mu=0
    // should check and give error if k<0
    // and max(y)/sigma > -1/k
    int N;
    N <- dims(y)[1];
    if (fabs(k) > 1e-15)
      return -(1+1/k)*sum(log1pv(y*k/sigma)) -N*log(sigma);
    else
      return -sum(y/sigma) -N*log(sigma); // limit k->0
  }
  vector gpareto_ccdf(vector y, real k, real sigma) {
    // generalised Pareto log ccdf with mu=0
    // should check and give error if k<0
    // and max(y)/sigma < -1/k
    if (fabs(k) > 1e-15)
      return exp((-1/k)*log1pv(y/sigma*k));
    else
      return exp(-y/sigma); // limit k->0
  }
}
  \end{lstlisting}
}
\end{frame}



\begin{frame}[fragile]

\frametitle{Other packages}

  \begin{itemize}
  \item R
    \begin{itemize}
    \item shinystan --- interactive diagnostics
    \item bayesplot --- visualization and model checking (see model checking in Ch 6)
    \item loo --- cross-validation model assessment, comparison and averaging (see Ch 7)
    \item projpred --- projection predictive variable selection
    \end{itemize}
    \vspace{\baselineskip}
  \item Python
    \begin{itemize}
    \item ArviZ --- visualization, and model checking and assessment (see Ch 6 and 7)
    \end{itemize}
  \end{itemize}

\end{frame}

\begin{frame}[fragile]

\frametitle{Different interfaces}

  \begin{itemize}
  \item RStan / PyStan
    \begin{itemize}
    \item C++ functions of Stan are called directly from R / Python
    \item Higher integration between R/Python and Stan, but maybe more
      difficult to install due to more requirements of compatible C++
      compilers and libraries
    \end{itemize}
  \item CmdStanR / CmdStanPy
    \begin{itemize}
    \item Lightweight interface on top of commandline program CmdStan
    \item Lacks some features that are not needed in this course, but
      is usually easier to install
    \end{itemize}
  \item More recent useful R packages
    \begin{itemize}
    \item posterior: for handling posterior draws, convergence diagnostics, and summaries
    \item tidybayes + ggdist: pretty plots
    \end{itemize}
  \end{itemize}

\end{frame}




\begin{frame}

\frametitle{Extra material for Stan}

  \begin{itemize}
  \item Andrew Gelman, Daniel Lee, and Jiqiang Guo (2015) Stan: A
    probabilistic programming language for Bayesian inference and
    optimization. \url{http://www.stat.columbia.edu/~gelman/research/published/stan_jebs_2.pdf}
  \item Carpenter et al (2017). Stan: A probabilistic programming
    language. Journal of Statistical Software
    76(1). \url{https://dox.doi.org/10.18637/jss.v076.i01}
  \item Stan User's Guide, Language Reference Manual, and Language
    Function Reference (in html and pdf)
    \url{https://mc-stan.org/users/documentation/}
    \begin{itemize}
    \item[-] easiest to start from Example Models in User's guide
    \end{itemize}
  \item Basics of Bayesian inference and Stan, part 1 Jonah Gabry \&
    Lauren Kennedy (StanCon 2019 Helsinki tutorial)
    \begin{itemize}
    \item[-]
      \url{https://www.youtube.com/watch?v=ZRpo41l02KQ&index=6&list=PLuwyh42iHquU4hUBQs20hkBsKSMrp6H0J}
    \item[-] \url{https://www.youtube.com/watch?v=6cc4N1vT8pk&index=7&list=PLuwyh42iHquU4hUBQs20hkBsKSMrp6H0J}
  \end{itemize}
  \end{itemize}
\end{frame}



%%%%%%%%%%%%%%%%%%%%%%%%%%%%%%%%%%%%%%%%%%%%%%%%%%%%%%%%%%%%%%%%%%


\end{document}
