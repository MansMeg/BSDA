%\documentclass[10pt,handout]{beamer}
\documentclass[10pt]{beamer}
\usepackage{babel} % Anpassa efter svenska. Ger svensk logga.
\usepackage[utf8]{inputenc} % Anpassa efter linux
\usepackage{graphicx}
\usepackage{hyperref}
\usepackage{listings}
%\input{../common/lststan} % Stan listing
\usepackage{lstbayes}
\usepackage[all,poly,ps,color]{xy}


\hypersetup{
    colorlinks=true,
    linkcolor=blue,
    filecolor=magenta,
    urlcolor=cyan,
}
\usepackage{../common/beamerthemeUppsala}
%\usetheme{Uppsala}
%\usecolortheme{UU} % Anpassa efter UU:s frger och logga
%\hypersetup{pdfpagemode=FullScreen} % Adobe Reader ska ppna fullskrm
\setbeamertemplate{itemize items}[circle]

% \usepackage{beamerthemesplit}
\usepackage{amsmath,amsfonts,amssymb}
% \usepackage{amssymb}
% \usepackage{graphics}
% \usepackage{graphicx}
% \usepackage{epsfig}
% \usepackage[latin1]{inputenc}
 \usepackage{color}
% \usepackage{fancybox}
% \usepackage{psfrag}
% \usepackage[english]{babel}
 \setbeamertemplate{footline}{\hfill\insertframenumber/\inserttotalframenumber}

% Input new commands
\input{../common/commands.tex}

\def\dashxy(#1){%
  /xydash{[#1] 0 setdash}def}
\def\grayxy(#1){%
  /xycolor{#1 setgray}def}
\newgraphescape{D}[1]{!{\ar @*{[!\dashxy(2 2)]} "#1"}}
\newgraphescape{P}[1]{!{\ar "#1"}}
\newgraphescape{F}[1]{!{*+=<2em>[F=]{#1}="#1"}}
\newgraphescape{O}[1]{!{*+=<2em>[F]{#1}="#1"}}
\newgraphescape{V}[1]{!{*+=<2em>[o][F]{#1}="#1"}}
\newgraphescape{B}[3]{!{{ "#1"*+#3\frm{} }.{ "#2"*+#3\frm{} } *+[F:!\grayxy(0.75)]\frm{}}}

%%%%%%%%%%%%%%%%%%%%%%%%%%%%%%%%%%%%%%%%%%%%%%%%%%%%%%%%%%%%%%%%%%

\setlength{\parskip}{3mm}
\title[]{{\color{black}Bayesian Statistics and Data Analysis \\ Lecture 9}}
\author[]{M{\aa}ns Magnusson \\ Department of Statistics, Uppsala University \\ Thanks to Aki Vehtari, Aalto University}
\date{}

\begin{document}

\frame{\titlepage
% \thispagestyle{empty}
}

%%%%%%%%%%%%%%%%%%%%%%%%%%%%%%%%%%%%%%%%%%%%%%%%%%%%%%%%%%%%%%%%%%


\section{Bayesian/Statistical Decision Theory}
\frame{\sectionpage}


\begin{frame}
\frametitle{Introduction to Decision Theory}

\begin{itemize}
  \item Statistical Decision Theory (SDT): Formalizing decision-making under uncertainty
  \pause
  \item Early work by Condorcet (1793-1794) and Dewey (1910)
%  \item Philosophical perspective: (Prerequisite for) \uured{rational decision-making}
  \item Two types of decision theory:
  \begin{enumerate}
      \item Normative (moral philosophy)
  \pause
      \item Descriptive (empirical)
  \pause
  \end{enumerate}
  \item SDT: Normative theory for rational decision-making
  \pause
  \item Although normative \uured{given a set goal}, i.e. not a moral philosophy
  \pause
  \item Balance "values" and information/knowledge
    \pause
  \item "Decision theory concerns goal-directed behaviour in the presence of options" (and under uncertainty) (Hansson, 1994).

\end{itemize}

\end{frame}

\begin{frame}
\frametitle{Bayesian decision theory}

  \begin{itemize}
  \item<+-> Potential decisions (or actions) $d \in D$
    \begin{itemize}
      \item $d$ may be categorical, ordinal, real, scalar, vector, etc.\\e.g. treat a person or not
      \pause
      \item fully in the control of the decision-maker
    \end{itemize}
  \pause
  \item<+-> Potential consequences (or outcomes) $x$
    \begin{itemize}
      \item $x$ may be categorical, ordinal, real, scalar, vector, etc.\\e.g. a person has covid-19
    \end{itemize}
    \pause
  \item<+-> Probability distribution of outcome given decision $p(x|d)$
    \begin{itemize}
    \item the decisions are controlled and thus $p(d)$ does not exist \pause
    \item Sometimes the decision has an effect in itself, hence $p(x|d)$\\
    e.g. $x$ outcome in exam, $d$ is whether to study or not
    \end{itemize}
  \end{itemize}

\end{frame}


\begin{frame}
\frametitle{Bayesian decision theory}

  \begin{itemize}
  \item<+->  Utility function $U(x, d)$ maps consequences to real value
    \begin{itemize}
      \item e.g. euro or expected lifetime
      \item instead of utility, sometimes cost or loss is defined as $-U(x, d)$
      \item can be multiple types of values
    \end{itemize}
  \pause
  \item \uured{Expected utility} for each decision decision $d$
  \[
  E[U(x, d)|d]=\int U(x, d) p(x|d) dx
  \]
  \pause
  \item<+-> \uured{Optimal decision}: $d^*$, which maximizes the expected utility (Maximal Expected Utility, MEU)
    \begin{equation*}
      d^*=\arg\max_d E[U(x, d)|d]
    \end{equation*}
  \end{itemize}

\end{frame}

\subsection{A Trivial Example}
\frame{\subsectionpage}

\begin{frame}
\frametitle{Example of decision making: 2 choices}

\begin{itemize}
\item Helen is going to pick mushrooms in a forest,  she notices a paw print which could made by a dog or a wolf
\pause
\item Helen measures that the length of the paw print is 14 cm
\item Based on data she want to infer the probability of wolf
\pause
\includegraphics[width=8cm]{figs/hatutus_likelihoods}
  observed length has been marked with a horizontal line
\item Likelihood of wolf is 0.92 (alternative being dog)
\end{itemize}

\end{frame}



\begin{frame}

\frametitle{Example of decision making}

  \begin{itemize}
  \item<+-> Helen also assumes that in her living area there are about one hundred times more free running dogs than wolves,
  \\ i.e. {\em a priori} probability for wolf, before observation is 1\%.
  \item Likelihood and posterior
    \begin{center}\leavevmode
      \begin{tabular}{| l | c c |}
        \hline
        Animal &  Likelihood & Posterior probability \\
        \hline
        Wolf     &  0.92            & 0.10      \\
        Dog    &  0.08        & 0.90    \\
        \hline
      \end{tabular}
    \end{center}
  \item<+-> \uured{Posterior probability} of wolf is 10\%
  \end{itemize}

\end{frame}

\begin{frame}

\frametitle{Example of decision making}

  \begin{itemize}
  \item<+-> Helen has to make decision whether to go pick mushrooms
  \item<+-> If she doesn't go to pick mushrooms utility is zero
  \item<+-> Helen assigns positive utility 1 for getting fresh mushrooms
  \item<+-> Helen assigns negative utility -1000 for a event that she goes to the forest and wolf attacks\\
  \item Utility function $U(d,x)$:

    \uncover<+->{
    \begin{minipage}[t]{58mm}
      \small
      \begin{tabular}{| l | c c |}
        \hline
        & \multicolumn{2}{ c |}{Animal $x$} \\
        Decision $d$           & Wolf & Dog \\
        \hline
        Stay home             & 0    & 0              \\
        Go to the forest      & -1000 & 1      \\
        \hline
      \end{tabular}\\
    \end{minipage}
    }
    \item Expected utilities\\
    \uncover<+->{
    \begin{minipage}[t]{58mm}
      \small
      \begin{tabular}{| l | c | }
        \hline
        & Conditional utility \\
        Action $d$        &  $E[U(x)|d]$ \\
        \hline
        Stay home         &  0       \\
        Go to the forest  &  -100+0.9 = -99.1     \\
        \hline
      \end{tabular}\\
    \end{minipage}
}
  \end{itemize}

\end{frame}

\begin{frame}

\frametitle{Example of decision making}

  \begin{itemize}
  \item \uured{Maximum likelihood} decision would be to assume that there is a wolf
  \pause
  \item \uured{Maximum posterior} decision would be to assume that there is a dog
  \pause
  \item \uured{Maximum Expected Utility} (MEU) decision is to stay home, even if it is more likely that the animal is dog
\vspace{5mm}
  \item The uncertainties (probabilities) related to all consequences need to be take into account
  \end{itemize}
\end{frame}


\section{Integrating Inference and Decisions}
\frame{\sectionpage}

\begin{frame}

\frametitle{Integrating inference and decisions}

  \begin{itemize}
    \item To make an optimal decision we need $p(x|d)$
    \pause
    \item In many situations we can approximate $p(x|d) \approx p(x)$
    \pause
    \item The benefit of Bayesian inference: \\
    We can use $p(x|d, y)$ i.e. integrating previous data in decision making (some times referred to as Bayesianism)
    \item Formal Bayesian Decision making hence have two parts:
    \begin{enumerate}
        \item Model $p(x|y,d)$ as good as possible \pause
        \item Define $U(x,d)$
    \end{enumerate}
  \end{itemize}

\end{frame}

\begin{frame}
\frametitle{Challenges with the utility function}

  \begin{itemize}
  \item Utility functions are rarely linear: \\How do we set it up?
  \pause
  \item What is the cost of human life/illness? Can it be formulated to a utility function?
  \pause
  \item Different parties might have different utilities (patient, physician, society)
  \pause
  \item Personal vs institutional decisions
  \begin{itemize}
     \item An individual have a subjective $p(x|d)$ and a subjective $U(x, d)$. Need for formal decision-making?
     \item An institution might be better suited. Decision-recommendations.
  \end{itemize}
  \item Decision theoretical approaches is better suited when $U(x,d)$ and $p(x|d)$ is well-defined.
  \end{itemize}
\end{frame}

\begin{frame}
\frametitle{Challenges with the probability model}

  \begin{itemize}
  \item We seldom know $p(x|d)$
  \pause
  \item We need to approximate: $\hat{p}(x|y,d) \approx p(x|d)$
  \pause
  \item If our approximation is bad - we will not necessarily make the optimal decision
  \item And we know that all models are wrong.
  \pause
  \item $\hat{p}(x|y,d)$ is a statistical problem. Can we trust the model?
  \item But it can also be extremely hard to estimate, say:
  \[
  P(\text{a new pandemic in 2025})
  \]
  \end{itemize}

\end{frame}

\begin{frame}

\frametitle{Challenges in decision making}

  \begin{itemize}
  \item Personal vs institutional decisions
  \begin{itemize}
     \item An individual have a subjective $p(x|d)$ and a subjective $U(x, d)$. Need for formal decision-making?
     \pause
     \item An institution might be better suited. Decision-recommendations.
  \end{itemize}
  \item Decision theoretical approaches is better suited when $U(x,d)$ and $p(x|d)$ is well-defined.
  \end{itemize}
\end{frame}

\begin{frame}

\frametitle{Multi-stage decision making}

  \begin{itemize}
  \item Slighly more complex: Multi-stage decision-making
  \item We need to take all uncertainties into account
  \item We can also condition after the decision is made
  \end{itemize}
\end{frame}


\begin{frame}
\frametitle{Multi-stage decision making example}

  \begin{itemize}
  \item 95-year-old has a tumor that is malignant with 90\% probability
  \item A priori knowledge
    \begin{itemize}
    \item<.-> expected lifetime is 34.8 months if no cancer
    \item<+-> expected lifetime is 16.7 months if cancer and radiation therapy is used
    \item<+-> expected lifetime is 20.3 months if cancer and surgery, but the probability of dying in surgery is 35\% (for 95 year old)
    \item<+-> expected lifetime is 5.6 months is cancer and no treatment
    \end{itemize}
  \item<+-> Which treatment to choose?
    \begin{itemize}
    \item<.-> quality-adjusted lifetime
    \item<.-> 1 month is subtracted from the time spent in treatments
    \end{itemize}
   \item<+-> Quality-adjusted lifetime
    \begin{itemize}
    \item<.-> Radiothreapy: 0.9*16.7 + 0.1*34.8 - 1 = 17.5mo
    \item<+-> Surgery: 0.35*0 + 0.65*(0.9*20.3 + 0.1*34.8 - 1) = 13.5mo
    \item<+-> No treatment: 0.9*5.6 + 0.1*34.8 = 8.5mo
    \end{itemize}
  \item<+-> Elaborated further in Bayesian Data Analysis
\end{itemize}

\end{frame}



\begin{frame}

\frametitle{Decisions under uncertainty}
%  \begin{center}
%    \includegraphics[width=5cm]{L1/figs/doc.jpg}
%  \end{center}
  Suppose a test is \uured{positive}. The probability of having covid-19 is: 0.095
  Lets put up the decision problem:
  \begin{enumerate}
      \item What is the decision set?
      \item We have: $p(x|d)$
      \item What is the Utility function $U(x,d)$
  \end{enumerate}
\end{frame}


\section{Course Evaluation}
\frame{\sectionpage}

\begin{frame}

\frametitle{Course Evaluation}

\begin{itemize}
\item What was good? What was fun?
\pause
\item What can be improved? What was annoying?
\pause
\item Did you get what you expected?
\pause
\item How can I get you to speak more during class?
\end{itemize}

\end{frame}


%%%%%%%%%%%%%%%%%%%%%%%%%%%%%%%%%%%%%%%%%%%%%%%%%%%%%%%%%%%%%%%%%%


\end{document}
